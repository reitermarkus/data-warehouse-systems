\documentclass[../paper.tex]{subfiles}


\begin{document}

In this section we provide a brief comparison about the pricing models and features,
provided by the four different DWaaS solutions discussed in section
\ref{sec:dwh-services-for-smes}.

One big advantage of DWaaS products is that service providers configure and manage
hardware and software resources, and therefore the customer only provides the data
and pays for the used services. The four different DWaaS products reviewed in this
paper offer four different pricing models. However all of them offer a free trial in
order to test their platform. Table \ref{tab:pricing-models} provides an overview
about the different pricing options for each DWaaS product.

\textit{Panoply} offers monthly and annual plans starting at \$200 per month. Their
pricing model is simple: users pay for the amount of data sources and storage space
they use. There are no extra costs for adding users or the number or queries run,
therefore costs are more predictable than with \textit{"on-demand"} models, like
\textit{Snowflake}. The pricing is based on data consumption per second, which means
users only pay for the compute and storage they use. This structure works well for
experienced users who know their average data consumption but this could be a huge
disadvantage for new users with little experience. Furthermore, \textit{Segment’s}
pricing tiers are based on the number of data sources and customers that are being
tracked. Furthermore \textit{Tableau} offers different packages, which are only billed
annually.

\begin{table}[]
    \centering
    \begin{tabular}{ l l l l l }
    \hline
    \multicolumn{5}{c}{Pricing}                                                                                                                                                                                         \\ \hline
    Panoply     & \multicolumn{4}{l}{\begin{tabular}[c]{@{}l@{}}- monthly and annual plans starting at \$200\\ -users pay for the number of data sources and amount of data\end{tabular}}                               \\ \hline
    Segment     & \multicolumn{4}{l}{\begin{tabular}[c]{@{}l@{}}- pricing tiers based on number of data sources and customer or visitors \\ - users pay for the number of data sources and amount of data\end{tabular}} \\ \hline
    Tableau     & \multicolumn{4}{l}{\begin{tabular}[c]{@{}l@{}}- different packages starting from \$12 per user per month\\ - billed annually\end{tabular}}                                                                                                                  \\ \hline
    Snowflake   & \multicolumn{4}{l}{\begin{tabular}[c]{@{}l@{}}- pricing based on data consumption per second\\ - works well for experienced users\end{tabular}}                                                       \\ \hline
    \end{tabular}
    \caption{Overview pricing models for Panoply, Segment, Tableau and Snowflake}\
    \label{tab:pricing-models}
\end{table}

In comparison to the other services, \textit{Snowflake} does not offer any data
analysis tools. It is designed as a centralised collection point, that distributes
the companies data to different destinations. Therefore the service is limited by
API calls.

The other services also include data analysis tools and therefore function as a central
collection hub for a companies data. Different plans are limited by data storage
and the amount of simultaneous data sources, that can be used.

Overall, not each service is ideally suited for all types of SMEs, yet the different
pricing tiers and their features cover a wide variety of them. From the review we
can conclude that all data warehouse services are considered suitable for an SME.

\end{document}
