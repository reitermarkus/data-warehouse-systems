\documentclass[../paper.tex]{subfiles}

\begin{document}

As the amount of digital data is ever increasing for all companies, the effort to manage it grows exponentially. To structure that data, most business have the option of implementing a data warehouse. Although there are many existing DWH implementations, only some are actually applicable to the setting of a smaller business. This is due to the fact that SMEs tend to lack certain expertise and have a limited budget, as well as a small amount of spare employees for IT. This further aggravates the use of many traditional DWH solutions, as they are tailored towards big enterprises and include many features, which are not relevant for a SME \cite{Raj2016}.

Furthermore, on-premise data warehouse systems require a certain level of storage and computing power, to fully utilise the advantages of the software. Those solutions often have a high up-front cost, making them a less ideal solution for a SME. Therefore, many vendors of DWH solutions offer their product as a cloud-based subscription service, which is well received by customers \cite{Agostino2013}.

Cloud-based data warehouses enable smaller business to fully utilise all needed features of the DWH technology, as they require considerably less time and expertise to set up and configure. Additionally, there is no up-front cost, as there is no local infrastructure required. Many popular vendors also offer a \textit{pay-as-you-go} subscription, which gives customers flexibility to try out their system for a very low fee. The reduction in cost and a static monthly payment model is the main reason why smaller companies can even consider implementing a DWH in their business process, as this was previously the main barrier, \cite{Fernandes2016}.

Business Intelligence is a kind of privilege which has been used from larger companies longer than a decade ago but during the last 10 years more and more SMEs choose to use this technology since it has been developed rapidly. These data analysis tools are now more lightweight and accessible for smaller businesses and therefore used to turn data into informed decisions in order to face main competitors. Small Business Analytics  is a technique and practice which measures a specific performance of a small company on an operational or strategic level. This technique is used on small datasets to gain insights on company processes. These insights can serve as key factors determining crucial decision-making processes. In most cases when organisations approach small data, they often overlook these insights. There are different reasons why small data should be treated seriously:
\begin{itemize}
	\item Focus on target - While big data sees the performance, small data is more focused on improving results. Key performance indicators (KPI) need to be identified and people should get one indicator assigned to track the development.
	\item Actionable - Big data serves information on every metric of each department. The problem is that all this data can get too general and overwhelming which forces data analysts to make strategic and organisational changes.
	\item All about what is happening now - Getting information from small data is quite easy and the data-source acts immediately compared to big data. If one needs a historical insight or wants to combine old data with current data, it is not possible without big data.
	\item Delivered ready to be served - Small data serves information in a strategic way which makes it easier to manage it and coworkers are more likely to utilise reports that will deliver clearer insights on the data.
\end{itemize}
Even though small data is a part of big data, both can be used separately, dependent on the quantity of departments in the business. Every business, even SMEs need a clear overview of where they stand on the market to achieve business goals.~\cite{Durcevic2018}

\end{document}
