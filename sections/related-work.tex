\documentclass[../paper.tex]{subfiles}

\begin{document}

As the amount of digital data is ever increasing for all companies, the effort to manage it grows exponentially. To structure that data, most business have the option of implementing a data warehouse. Although there are many existing DWH implementations, only some are actually applicable to the setting of a smaller business. This is due to the fact that SMEs tend to lack certain expertise and have a limited budget, as well as a small amount of spare employees for IT. This further aggravates the use of many traditional DWH solutions, as they are tailored towards big enterprises and include many features, which are not relevant for a SME \cite{Raj2016}.

Furthermore, on-premise data warehouse systems require a certain level of storage and computing power, to fully utilise the advantages of the software. Those solutions often have a high up-front cost, making them a less ideal solution for a SME. Therefore, many vendors of DWH solutions offer their product as a cloud-based subscription service, which is well received by customers \cite{Agostino2013}.

Cloud-based data warehouses enable smaller business to fully utilise all needed features of the DWH technology, as they require considerably less time and expertise to set up and configure. Additionally, there is no up-front cost, as there is no local infrastructure required. Many popular vendors also offer a \textit{pay-as-you-go} subscription, which gives customers flexibility to try out their system for a very low fee. The reduction in cost and a static monthly payment model is the main reason why smaller companies can even consider implementing a DWH in their business process, as this was previously the main barrier, \cite{Fernandes2016}.

Business Intelligence (BI) is a systematic approach that compromises a set of tools and guidelines, which helps a business to analyse its operations and report on different statistics. Systems that offer BI features, are use a data warehouse for analytical queries. While this is a common approach within global enterprises for years, SMEs rarely adopt any form of business intelligence, since the lack the expertise and capabilities to deploy systems for it ~\cite{Golfarelli2004}.

SMEs therefore require a simple and streamlined approach, as they are not able to analytically deal with big data in their existing systems. With the addition of cloud computing, SMEs no longer need complex hardware and software systems, which have high maintenance costs, to deal with big data. \textit{Vajjhala and Ramollari} argue that cloud services also encapsulate some of the complexity, giving SMEs the chance the acquire flexible computing power for data warehouses to use business intelligence tools ~\cite{Vajjhala2016}.

\end{document}
