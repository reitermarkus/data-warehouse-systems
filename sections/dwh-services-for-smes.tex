\documentclass[../paper.tex]{subfiles}

\begin{document}

For small and medium businesses, probably the most important aspect when
choosing a data warehouse system is cost, both for the initial development and
for the ongoing maintenance of such a system.

Nowadays, Software as a Service (SaaS) can provide many advantages over
traditional services. The pay-as-you-go model is very friendly towards small
businesses which could not otherwise easily justify the upfront cost for
servers and related costs for hosting a data warehouse.

This means that, in many cases, SaaS is the most cost-effective and also the
simplest solution for small businesses to opt for. When comparing them to
traditional services, SaaS products virtually don't need any setup time and
can be deployed instantly.

In the case of data warehouse systems, SaaS is also commonly referred to as
Data Warehouse as a Service (DWaaS).

Given the advantages above, we focus in this section on some concrete DWaaS
products and review what they have in common, how they differ and whether they
are in fact suitable for small businesses.

\subsection{Segment (by Twilio)}

After first logging into Segment, the user can choose the team they are working on (Engineering, Marketing, Founder/Executive, Product, Analytics) and select the first data source, e.g. a website, programming language or HTTP API. Next, data destinations have to be selected, e.g. Google Analytics, Intercom, etc. Finally, the user gets to the dashboard, which provides an overview of all data sources and destinations and a way to add new ones.

In total, Segment supports 98 different data sources and 650 data destinations at the time of writing. Additionally, creating custom data sources and destinations by building JavaScript function that access the corresponding API. Also, by supporting programming languages as data source and webhooks as data destinations, virtually any software can be integrated.

Segment does not offer any Analytics capability on its own but is meant to simplify data collection and distribution by managing all data sources and destinations in a single place, therefore reducing complexity and increasing flexibility. For example, website analytics can be switched from Google Analytics~\cite{google_analytics} to GoSquared~\cite{gosquared} without changing the website itself.

\subsection{Panoply}

\textit{Panoply} is a data warehouse solution build on \textit{AWS Redshift} and offers four plans: \textit{LITE}, \textit{STARTER}, \textit{PRO} and \textit{BUSINESS}. The last of them is specifically aimed at SMBs, according to their own website. There is a free version available for testing which has the functionality if the \textit{LITE} plan for a period of 14 days. After logging in for the first time, the website prompts a user to create a data warehouse, that is required to have a unique name. Next, a user is prompted with the possibility of adding a data source, which can also be skipped. Afterwards, the user has full access to the instance.

\textit{Panoply} offers the integration of 122 data sources that have been integrated by the Panoply team. Additionally, there are 131 data sources, which are developed by partners, that can be added. In total, 253 different data source are supported. To analyse the collected data, \textit{Panoply} offers the integration of 43 visualisation tools, of which 42 are \textit{BI} tools.

\textit{Panoply} is a full solution to sync, store and access a companies data, while also providing analytic features. Additionally to the supported visualisation tools, data can be structured and viewed in a traditional tabular form.

The different pricing tiers are differentiated by three main parts: amount of data sources, storage space and support. The suggested SMB solution, called the \textit{BUSINESS} plan, includes 10 distinct data sources, 100 GB of storage and support with a reaction time of less than an hour. It also includes \textit{Data Governance} features, yet the storage itself is based in the \textit{USA}, without any other option. All plans offer an unlimited amount of users. More storage and data sources is possible for the \textit{Enterprise} plan, which is adapted to a companies needs. This adds the possibility of storing the data in one of 19 different countries.

\end{document}
