\documentclass[../paper.tex]{subfiles}

\begin{document}

A \textit{data warehouse} (\textit{DWH}) is a special type of database system, that focuses on reporting and analysing of it's data. Implementing such a system reduces the complexity to access business data in an analytical way and is an important step to achieve \textit{business intelligence (BI)}. Enterprises typically bundle the data of all their operational databases together in one data warehouse. All departments within a company still use their own database for day to day production, as they don't use it for analysing and reporting of their business process. Hence the DWH system is mostly used in the management layer of an enterprise, since they deal with internal reporting and business analytics.

With increasing digitisation of business processes and communication, the amount of data, that companies are collecting, is increasing rapidly across all business sectors. Therefore a data warehouse system is becoming more interesting for many small- and medium-sized enterprises (\textit{SMEs}), as they require a systematic approach to analyse their business data in a productive way. In the context of this research, companies with at most 250 employees are considered a SME, also known as \textit{small and medium business (SMB)}.

The increasing demand for such systems, has lead to an increase in development for specific data warehouse solutions targeting SMEs. This paper aims to compare and analyse such systems for their suitability in the context of a small or medium enterprise. Furthermore, a comparison of different systems is used to give a general baseline for a lightweight implementation of a DWH in a SME.

\end{document}
