\documentclass[../paper.tex]{subfiles}

\begin{document}

A data warehouse (DWH) is a special type of database system that focuses on
reporting and analysis of its data. Implementing such a system reduces the
complexity to access business data in an analytical way and is an important
step to achieve business intelligence (BI). Enterprises typically bundle the
data of all their operational databases together in one data warehouse. All
departments within a company still use their own database for day to day
production, as they don't use it for analysing and reporting of their business
process. Hence, the DWH system is mostly used in the management layer of an
enterprise which deals with internal reporting and business analytics.

With increasing digitisation of business processes and communication, the
amount of data that companies are collecting is increasing rapidly across all
business sectors. Therefore, a data warehouse system is becoming more
interesting for many small- and medium-sized enterprises (SMEs) as they require
a systematic approach to analyse their business data in a productive way. In
the context of this research, companies with at most 250 employees are
considered a SME, also known as small and medium business (SMB).

The increasing demand for such systems has lead to an increase in development
for specific data warehouse solutions targeting SMEs. This paper aims to
compare and analyse such systems for their suitability in the context of a
small or medium enterprise. Furthermore, a comparison of different systems is
used to give a general baseline for a lightweight implementation of a DWH in a
SME. The work also tries to highlight the essential features a data warehouse
is required to have if being applied in an SME. Additionally, the review
process of existing services is thoroughly described, to achieve a high
reproducibility of the results.

\Cref{sec:related-work} discusses some related work and gives an overview of
previous approaches. Afterwards, \cref{sec:dwh-needs-for-smes} presents some
factors of success for a data warehouse implementation at an SME and highlights
the specific needs for such a use case. The methodology for the review of DWH
services is described in \cref{sec:dwh-services-for-smes}. Finally, a
conclusion and outlook for some future work is presented in
\cref{sec:conclusion}.

\end{document}
