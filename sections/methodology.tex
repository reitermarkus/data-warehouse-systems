\documentclass[../paper.tex]{subfiles}

\begin{document}

The conclusion about the current state of data warehouse solutions for SMEs is drawn from the several service evaluations. To achieve a high degree of reproducibility, this evaluation and testing has been done by a fixed methodology that thoroughly describes each step. As the review of those services has been conducted by multiple researchers, a predefined evaluation approach was necessary to collect the same type of data and test for a fixed set of use-cases.

Upon the selection of a service to review, an account is created to check if a free trial version is even available. If this was not the case, the service was excluded from the review, since just evaluating the specifications and documentation was not seen as sufficient enough to reason about the suitability for an SME.

The initial setup process of an account and the data warehouse is also discussed within the review, since the whole process should not be too complex for IT staff that are not familiar with the underlying technology of a data warehouse. This is due to the fact, that such highly specialised staff might not existing in an SME needing a simple data warehouse service solution.

Afterwards, the review focused on the features of the service. Therefore, the following questions were asked for each:
\begin{itemize}
  \item How many different data sources are supported?
  \item Dose the service include analytical features?
  \begin{itemize}
    \item If not, how many data destinations are supported?
    \item If it does, how many analytical visualisation tools are supported?
  \end{itemize}
  \item What are the different pricing tiers and which one is recommended for SMEs?
\end{itemize}

In addition to those questions, any other information like underlying services or legal certifications were noted within a services review. Furthermore, the overall design of the services interface is rated for usability.

While those aspects are mostly very objective, some subjective bias by the researcher cannot be excluded, especially for the initial setup of a service. The complexity of the service and its interface itself is in generally a completely subjective perception and therefore might have been different for other researchers. However this is not seen as a major problem, since the core suitability of a service for an SME is determined by its feature set and price.

\end{document}