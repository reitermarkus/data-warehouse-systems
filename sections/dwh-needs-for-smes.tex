\documentclass[../paper.tex]{subfiles}

\begin{document}

The data generated and captured by an SME is its most important asset. Since
the amount of available data is constantly growing, the only solution to avoid
data management problems is to use a dedicated data warehouse (DWH) system. Not
every SME may need such a storage solution right from the start, but there are
some common signs which show a business when it would be more efficient to
switch over or start with a data warehouse system.

Heavy reliance on spreadsheets is for example one critical sign why SMEs should
use a DWH System. Spreadsheets are very commonly used in pretty much every
business and its different departments to track data. While in most cases they
seem to be pretty universal, a lot of these spreadsheets can grow to immense
sizes and can become unmanageable. Combining these large files manually to
create a report takes a lot of time, not to mention the fact that every
department may also rely on different spreadsheets.

Spreadsheets are designed to take a specific amount of data divided into rows
and columns. Continuous adding of data can lead to “spreadsheet overwhelming”.
The file itself can become either sluggish or prevent the user from adding rows
and/or columns altogether. Therefore, a data warehouse system can definitely
increase the productivity, especially when combining multiple spreadsheets.

If employees in different departments work on these spreadsheets and one person
needs to wait on specific information to create a report or analyse data, this
takes too much time just to wait on other employees. With a DWH on the other
hand, data gets added directly into one centralised data location and analysis
can be performed at any minute.

Discrepancies in data and reports can be the result of different departments
creating their own data and reports. The difference in the results can be time
consuming to sort out and for SMEs this can lead to costly mistakes. In most
cases, the reason is caused by adding different, sometimes untrustworthy data
sources. If the point of data discrepancy is reached, it may be time to remedy
this problem by looking into a data warehouse system which ensures mistakes
like duplicate data are eliminated.

If the time invested in creating reports is too much, then SMEs should decide
on using a DWH System. Ideally, such reports can be created with a few clicks
and prevent employees from going to different sources to check if the data is
already updated. Since data warehouses consolidate data, all departments can
rely on a single source for data. Maintenance can be further simplified by
using the ability of such systems to set up automatic updates if the source
data gets changed or updated in order to guarantee the data which departments
rely on is always correct.

\end{document}
