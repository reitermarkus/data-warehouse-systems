\documentclass[../paper.tex]{subfiles}

\begin{document}

The data generated and captured by an \textit{SME} is the most important asset available for the company itself. Since the amount of available
data is constantly growing the only solution to not get into data management problems is to use a specific \textit{Data Warehouse System (DWH).}
It might not seem that every \textit{SME} needs such a storage solution from the very first minute but there are different signs which show a business
why it would be more efficient to switch over or start with a data warehouse system.

\textit{Heavy reliance on spreadsheets} is for example one critical sign why \textit{SMEs} should use a DWH System. The spreadsheet itself is a very
common used file type in pretty much every business and its different departments to track data. While in most cases it seems to be pretty universal,
a lot of these spreadsheets can grow to immense size and can become unmanageable. Combining the fact of growing sheets across all departments, combining
these files to create a manual report takes a lot of time, not to mention the fact that every department can also rely on different sheets.

Spreadsheets are designed to take a specific amount of data divided into rows and columns. Repetitive data adding can lead to
\textit{spreadsheet overwhelming}. The file itself can handle either sluggish or just prevent the user from adding rows and/or columns. Therefore a
data warehouse system can definitely increase the productivity, especially if multiple different sheets get combined.

If employed in different departments work on these sheets and one person needs to wait on specific information to create a report or 
analyse data, it \textit{takes too much time just to wait} on other employees. If the data person needs to create a report gets added directly into one
business centralised data source, analysing can be done at any minute. Furthermore other members in the same department also don't have to wait for
data, due to an employee being too busy at the moment.

\textit{Discrepancies in data and reports} can be the result of different departments creating their own data and reports. The difference in the results
can be time consuming to sort out and for \textit{SMEs} this can lead to costly mistakes. In most cases the reason is caused by adding different,
sometimes not trustworthy data sources. If the point of data discrepancy is reached it may be time for businesses to sort out this problem by looking
into a data warehouse system which ensures eliminating mistakes like duplicate data.

If the \textit{time invested in creating reports} is too much, then \textit{SMEs} should decide using a DWH System. Ideally such reports can be created
with few clicks and prevent employees from going to different sources to check if the data is already updated. Since data warehouses consolidate data,
all departments have to just turn to one source for data. Maintenance can be further simplified by using the ability of such systems to set up to
automatically update if the source data gets changed or updated and it is guaranteed that the data which departments rely on is always correct.

\end{document}
